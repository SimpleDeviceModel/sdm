\usepackage{microtype}
\usepackage{alltt}
\usepackage{enumitem}
\usepackage{multirow}
\usepackage{amsmath}
\usepackage{needspace}
\usepackage{longtable}

\usepackage{tcolorbox}
\tcbuselibrary{breakable}

\usepackage{fontspec}
\setmainfont[Ligatures=TeX]{XCharter}
\setmonofont[Scale=MatchLowercase]{DejaVu Sans Mono}

\usepackage[english]{babel}

\usepackage{listings}
\lstset{breaklines,breakatwhitespace,showstringspaces=false,tabsize=4}

\usepackage{perpage}
\MakePerPage{footnote}

\newcommand{\styledtitleref}[1]{\emph{\titleref{#1}}}
\newcommand{\lscaps}[1]{\textls[25]{\MakeTextUppercase{\textit{#1}}}}

\newlength{\baseparindent}
\setlength{\baseparindent}{\parindent}

%%%%%%%%%%%%%%%%%%%%%%%%%%%%%%%%%%
% Various Memoir class settings
%%%%%%%%%%%%%%%%%%%%%%%%%%%%%%%%%%

\sloppybottom
\setsecnumdepth{subsection}

\makeevenhead{headings}{\thepage}{}{\itshape\lscaps{\leftmark}}
\makeoddhead{headings}{\itshape\lscaps{\rightmark}}{}{\thepage}

\setpnumwidth{2.55em}
\setrmarg{3.55em}

%%%%%%%%%%%%%%%%%%%%%%%%%%%%%%%%%%
% Shell commands
%%%%%%%%%%%%%%%%%%%%%%%%%%%%%%%%%%

\newcommand{\shellcmd}[1]{\texttt{#1}}
\newcommand{\smallshellcmd}[1]{\small\texttt{#1}}

\newenvironment{shellcmds}{%
	\vspace{0.5\baselineskip}\begin{tcolorbox}[fontupper=\footnotesize]\begin{alltt}%
}{%
	\end{alltt}\end{tcolorbox}\vspace{0.5\baselineskip}%
}

\newenvironment{breakshellcmds}{%
	\vspace{0.5\baselineskip}\begin{tcolorbox}[fontupper=\footnotesize,breakable]\begin{alltt}%
		}{%
	\end{alltt}\end{tcolorbox}\vspace{0.5\baselineskip}%
}

\newenvironment{breakshellcmdsnamed}[1]{%
	\vspace{0.5\baselineskip}\begin{tcolorbox}[fontupper=\footnotesize,breakable,title=#1]\begin{alltt}%
		}{%
	\end{alltt}\end{tcolorbox}\vspace{0.5\baselineskip}%
}

%%%%%%%%%%%%%%%%%%%%%%%%%%%%%%%%%%
% Progam elements
%%%%%%%%%%%%%%%%%%%%%%%%%%%%%%%%%%

\newcommand{\uielement}[1]{\emph{#1}}

%%%%%%%%%%%%%%%%%%%%%%%%%%%%%%%%%%
% Tables
%%%%%%%%%%%%%%%%%%%%%%%%%%%%%%%%%%

\newcolumntype{L}{>{\raggedright\arraybackslash}X}
\newcolumntype{R}{>{\raggedleft\arraybackslash}X}
\newcolumntype{C}{>{\centering\arraybackslash}X}

\newcolumntype{Q}[1]{>{\raggedright\arraybackslash}p{#1}}
\newcolumntype{E}[1]{>{\raggedleft\arraybackslash}p{#1}}
\newcolumntype{W}[1]{>{\centering\arraybackslash}p{#1}}

\newcommand{\tabcutin}[2]{\multicolumn{#1}{c}{\emph{#2}}}

% Declare esktlongtable environment

\RequirePackage{longtable}
\RequirePackage{caption}

\setlength\LTleft{0pt}
\setlength\LTright{0pt}

\DeclareCaptionLabelFormat{onlynumber}{#2}
\DeclareCaptionFormat{conttable}{Continuation of the Table #1}

\newenvironment{eskdlongtable}[4]{%
	\noindent\begin{longtable}{@{\extracolsep{\fill}}#1}
	\ifx \\#3\\ \else \label{#3}\\ \fi%
	\caption{#2}\\
	\toprule
	#4\\
	\midrule\endfirsthead
	\captionsetup{labelformat=onlynumber,format=conttable}\caption[]{#2}\\
	\toprule
	#4\\
	\midrule\endhead
}{%
	\end{longtable}
}

%%%%%%%%%%%%%%%%%%%%%%%%%%%%%%%%%%
% Code
%%%%%%%%%%%%%%%%%%%%%%%%%%%%%%%%%%

\newcommand{\expr}[1]{\texttt{#1}}
\newcommand{\cexpr}[1]{\texttt{\lstinline[language=C++]`#1`}}
\newcommand{\luaexpr}[1]{\texttt{\lstinline[language={[5.2]Lua}]`#1`}}

\lstnewenvironment{ccode}{\lstset{language=C++}\ttfamily}{}
\lstnewenvironment{luacode}{\lstset{language=[5.2]Lua}\ttfamily}{}

\newcommand{\objtype}[1]{\emph{#1}}
\newcommand{\luatype}[1]{\emph{#1}}

% Function description

\newlength{\funcparskip}
\setlength{\funcparskip}{0.5\baselineskip}

\lstnewenvironment{cfuncprototype}{%
	\needspace{4\baselineskip}
	\lstset{language=C++,aboveskip=1\funcparskip,belowskip=2\funcparskip,basicstyle=\ttfamily}%
}{}

\newenvironment{luafuncprototype}{
	\par\vspace{0.5\funcparskip}%
	\ttfamily\bfseries\noindent\raggedright\ignorespaces%
}{%
	\par\nopagebreak%
	\vspace{1.5\funcparskip}\nopagebreak
}

\newenvironment{funcdescr}
{\setlength{\parindent}{0pt}}
{\par\vspace{\funcparskip}}

\newcommand{\funcparam}[1]{\item\expr{#1}}

\newenvironment{funcparams}{%
	\noindent \emph{Parameters}:\par\nopagebreak%
	\vspace{0.5\funcparskip}\nopagebreak%
	\begin{itemize}[topsep=0pt,partopsep=0pt,parsep=0pt,itemsep=0.25\funcparskip,label=,leftmargin=\parindent]
}{%
	\end{itemize}
	\par\vspace{\funcparskip}
}

\newenvironment{funcret}{%
	\noindent \emph{Return value}:\par\nopagebreak%
	\vspace{0.5\funcparskip}\nopagebreak%
}{%
	\par\vspace{\funcparskip}
}

\newenvironment{funcremarks}{%
	\noindent \emph{Remarks}:\par\nopagebreak%
	\vspace{0.5\funcparskip}\nopagebreak%
}{%
	\par\vspace{\funcparskip}
}

\newenvironment{funcexamples}{%
	\noindent \emph{Examples}:\par\nopagebreak%
	\vspace{0.5\funcparskip}\nopagebreak%
}{%
	\par\vspace{\funcparskip}
}

%%%%%%%%%%%%%%%%%%%%%%%%%%%%%%%%%%
% Various stuff
%%%%%%%%%%%%%%%%%%%%%%%%%%%%%%%%%%

\newcommand{\cplusplus}{C\texttt{\raisebox{0.05em}{++}}}

\newcommand{\tocitem}[2]{\phantomsection\addcontentsline{toc}{#1}{#2}}

%%%%%%%%%%%%%%%%%%%%%%%%%%%%%%%%%%
% Hyperlinks
%%%%%%%%%%%%%%%%%%%%%%%%%%%%%%%%%%

\usepackage[bookmarks=true,
	bookmarksnumbered=true,
	bookmarksdepth=3,
	hypertexnames=false]
	{hyperref}
